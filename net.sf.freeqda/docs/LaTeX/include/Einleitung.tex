\chapter{Einleitung}

Qualitative Forschung ist ein wesentlicher Bestandteil der internationalen Pflegeforschung,
insbesondere, wenn es um die Erforschung von Erfahrungen in Gesundheit und Krankheit von
Individuen oder deren sozialem Umfeld geht \cite{lobiondo05}. Insbesondere im
Zuge der Mixed-Methods-Diskussion in der U.S.-amerikanischen Sozialforschung, die ebenfalls
schon spürbare Auswirkungen auf den Bereich der Gesundheits- und Pflegeforschung hat, rücken
qualitative und quantitative Forschungsmethoden noch näher in den gemeinsamen Fokus der
Wissenschaftler und Forscher \cite{teddlie08}. Vor diesem Hintergrund zeichnen sich
neue Herausforderungen im Bereich der computergestützten qualitativen Datenanalysesoftware
(CAQDAS) ab \cite{kuckartz07}. So existieren bereits einige unterschiedliche Analyseprogramme,
die sich auch für den Bereich der Pflegeforschung bewährt haben, jedoch liegt die Anwenderlogik
derjenigen Fachdisziplin zugrunde, aus der sie heraus entwickelt wurde. Dies macht sich
insbesondere in der praktischen Handhabung, der Logik, der Nachvollziehbarkeit und fehlender
Programmeigenschaften bemerkbar \cite{lewins07, helfferich05}. Die bekannteste
Software ist derzeit MaxQDA (\url{http://www.maxqda.de}), deren Einsatz kostenpflichtig (derzeit ca.
1.000 Euro pro Arbeitsplatz) und auf das Betriebssystem Windows beschränkt ist. Die Lizenzkosten
können gerade für Nachwuchswissenschaftler und/oder Studierende ein Hindernis darstellen,
welches qualitative Forschungs- oder Qualifikationsarbeiten erschwert und/oder einschränkt. Dies
gilt einerseits für Deutschland, aber auch für Pflegeforschung in strukturschwachen Ländern wie
beispielsweise Litauen oder die Slowakei.
Durch die kommerzielle Vermarktung ist der Quellcode der Software nicht einsehbar und kann 
daher nicht erweitert, verändert oder auf persönliche, fachspezifische oder projektbezogene Anforderungen angepasst werden. 
Dies trifft auch auf die von MaxQDA generierten Projektdateien zu, die einen Export des Arbeitsmaterials zur weiteren Nutzung 
auf anderen Systemen ausschließen.
